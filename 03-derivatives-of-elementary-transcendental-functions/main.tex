\documentclass[12pt]{article}

\usepackage[margin=1in]{geometry}
\usepackage{xcolor, color, soul}
\usepackage{amssymb}
\usepackage{enumitem}
\usepackage{booktabs}
\usepackage{amsmath}
\usepackage{array}
\usepackage{tabularx}
\usepackage{graphicx}
\usepackage{wrapfig}
\usepackage{blindtext}

\graphicspath{ {./images/} }
\sethlcolor{yellow}

\newcommand{\class}{Derivatives of Elementary Transcendental Functions}
\newcommand{\datewritten}{Term 1, Fall '23}
\newcommand{\instructor}{Course Instructor: Noel T. Fortun}
\newcommand{\notes}{A.L. Maagma}
\newcommand{\follow}{\bigskip\noindent}
\newcommand{\spaces}{\quad\quad\quad}
\newcommand{\spacee}{\quad}
\newcommand{\point}{\,\cdot\,}

\newcommand{\mins}{-}
\newcommand{\block}[1]{\[{#1}\]}
\newcommand{\inline}[1]{\({#1}\)}
\newcommand{\proving}[1]{\begin{align*}{#1}\end{align*}}
\newcommand{\numbers}[1]{\begin{enumerate}[label*=\arabic*.]{#1}\end{enumerate}}
\newcommand{\enclose}[1]{\fbox{\parbox{\dimexpr\linewidth-2\fboxsep-2\fboxrule}{{#1}}}}

\newenvironment{conditions}
{\par\vspace{\abovedisplayskip}\noindent\begin{tabular}{>{$}l<{$} @{${}={}$} l}}
{\end{tabular}\par\vspace*{\belowdisplayskip}}

% ---------------- Start Document ----------------

\begin{document}
\pagestyle{plain}
\thispagestyle{empty}

% -------------------- Header --------------------

\noindent
\begin{tabular*}{\textwidth}{l @{\extracolsep{\fill}} r @{\extracolsep{6pt}} l}
    \textbf{\class} && \textbf{Date: \datewritten} \\
    \textbf{\instructor} && \textbf{Notes: \notes} \\
\end{tabular*}\\
\rule[2ex]{\textwidth}{2pt}

% --------------------- Body ---------------------

\noindent\section{Derivatives of Elementary Transcendental Functions}

    \enclose{
        \textbf{DEFINITION.}
        In general, the term \hl{transcendental} means non-algebraic.
        A transcendental function is a function that is not expressible as a finite combination of the algebraic opperations of addition, subtraction, multiplication, division, raising to a power, and extracting a root.
        An example includes the function \inline{\log{x}}, \inline{\sin{x}}, \inline{\cos{x}}, \inline{e^x}, and any functions containing them.
    }

\follow\section{The Natural Logarithmic Function}

    \enclose{
        \textbf{DEFINITION.}
        The function defined by \block{f (x) = \log_{e}{x} = \ln{x}} \inline{(x > 0, e \approx{} 2.718281 \dots)} is called the \hl{natural logarithmic function}.
        
        \follow\textbf{NOTE.}
        The equation \inline{y = \ln{x}} is equivalent to \inline{e^y = x}.
    }

    \follow\subsection{Logarithmic Properties}

        \enclose{
            \textbf{DEFINITION.}
            If \textit{a} and \textit{b} are positive numbers and \textit{n} is rational, then the following properties are true.
            \begin{enumerate}
                \setlength\itemsep{0em}
                \item \inline{\ln{1} = 0}; if \inline{x > 1}, then \inline{y = \ln{x} > 0}, else vice versa.
                \item \inline{\ln{(ab)} = \ln{a} + \ln{b}}
                \item \inline{\ln{(a^n)} = n \ln{a}}
                \item \inline{\ln{(\frac{a}{b})} = \ln{a} \mins{} \ln{b}}
            \end{enumerate}
        }
    
    \follow\subsection{Definition of the Natural Logarithmic Function}

        \enclose{
            \textbf{DEFINITION.}
            The natural logarithmic function is defined by \block{\ln{x} = \int_{1}^{x} \frac{1}{t} \, dt, x > 0}
            The domain of the natural logarithmic function is the set of all positive real numbers.
        }

    \follow\subsection{Derivative of the Natural Logarithmic Function}

        \enclose{
            \textbf{DEFINITION.}
            Let \textit{u} be a differentiable function of \textit{x}.
            \begin{enumerate}
                \setlength\itemsep{0em}
                \item \inline{\frac{d}{dx} [\ln{x}] = \frac{1}{x}, x > 0}
                \item \inline{\frac{d}{dx} [\ln{u}] = \frac{1}{u} \frac{du}{dx} = \frac{u'}{u}, u > 0}
            \end{enumerate}
        }

        \follow\textbf{EXAMPLE 2.3.1.} 
        Solve for the equation \inline{\frac{d}{dx} [\ln{2x}]}.
        \proving{
            &= \frac{1}{2x}
        }

        \follow\textbf{EXAMPLE 2.3.2.}
        Solve for the equation \inline{\frac{d}{dx} [\ln{(x^2 + 1)}]}.
        \proving{
            &= \frac{1}{u} \frac{du}{dx}               & u = x^2 + 1 \\
            &= \frac{1}{x^2 + 1} \frac{2x dx}{dx}      & du = 2x \, dx \\
            &= \frac{2x}{x^2 + 1}
        }

        \follow\textbf{EXAMPLE 2.3.3.}
        Solve for the equation \inline{\frac{d}{dx} [x \ln{x}]}.
        \proving{
            &= x \point{} \frac{1}{x} \\
            &= 1
        }

        \follow\textbf{EXAMPLE 2.3.4.}
        Solve for the equation \inline{\frac{d}{dx} [{(\ln{x})}^3]}.
        \proving{
            &= 3 (\ln{x})^2 \point{} \frac{1}{x} \\
            &= \frac{3 (\ln{x})^2}{x}
        }

        \follow\textbf{EXAMPLE 2.3.5.}
        Solve for the equation \inline{y = \ln{[(4x^2 + 3) (2x \mins{} 1)]}}.
        \proving{
            y &= \ln{(4x^2 + 3)} + \ln{(2x \mins{} 1)} \\
            \frac{dy}{dx} &= \frac{8x}{4x^2 + 3} + \frac{2}{2x \mins{} 1}
        }

        \follow\textbf{EXAMPLE 2.3.6.}
        Solve for the equation \inline{y = \ln{\left(\frac{x}{x + 1}\right)}}.
        \proving{
            y &= \ln{(x) {(x + 1)}^{-1}} \\
            y &= \ln{(x)} \mins{} \ln{(x + 1)} \\
            \frac{dy}{dx} &= \frac{1}{x} \mins{} \frac{1}{x + 1}
        }

        \follow\textbf{EXAMPLE 2.3.7.}
        By implicit differentiation, find the dy/dx of \inline{\ln{(\frac{x}{y})} + xy = 1}.
        \proving{
            \ln{(x)} \mins{} \ln{(y)} + xy &= 1                                                                     & f (y) &= f (y (x)) \\
            \frac{d}{dx} \left(\ln{(x)} \mins{} \ln{(y)} + xy\right) &= 1                                           & &= \ln{y} \\
            \frac{1}{x} \mins{} \frac{1}{y} \frac{dy}{dx} + y + x \frac{dy}{dx} &= 0                                & &= \ln{[y (x)]} \\
            \frac{dy}{dx} \left(x \mins{} \frac{1}{y}\right) &= \mins{} y \mins{} \frac{1}{x}                       & &= \frac{1}{y} \point{} y' = \frac{1}{y} \frac{dy}{dx} \\
            \frac{dy}{dx} &= \left(\frac{\mins{} xy \mins{} 1}{x}\right) \left(\frac{y}{xy \mins{} 1}\right) \\
            \frac{dy}{dx} &= \frac{\mins{} xy^2 \mins{} y}{x^2y \mins{} x}
        }

        \follow\textbf{EXAMPLE 2.3.8.}
        Solve for the equation \inline{\ln{(x + y)} \mins{} \ln{(x \mins{} y)} = 4}.
        % \proving{
        %     \frac{1}{x + y} \left(1 + \frac{dy}{dx}\right) \mins{} \frac{1}{x \mins{} y} \left(1 \mins{} \frac{dy}{dx}\right) &= 0 \\
        %     \frac{1}{x + y} + \frac{1}{x + y} \frac{dy}{dx} \mins{} \frac{1}{x \mins{} y} + \frac{1}{x \mins{} y} \frac{dy}{dx} &= 0 \\
        %     \frac{1}{x + y} \frac{dy}{dx} + \frac{1}{x \mins{} y} \frac{dy}{dx} &= \frac{1}{x \mins{} y} \mins{} \frac{1}{x + y} \\
        %     \frac{dy}{dx} \left(\frac{1}{x + y} + \frac{1}{x \mins{} y}\right) &= \frac{1}{x \mins{} y} \mins{} \frac{1}{x + y} \\
        %     \frac{dy}{dx} &= \frac{x^2 \mins{} y^2}{x \mins{} y} \mins{} \frac{x^2 \mins{} y^2}{x + y}
        % }

\follow\section{Logarithmic Differentiation}

    \enclose{
        \textbf{DEFINITION.}
        It is sometimes convenient to use logarithms as aids in differentiating non-logarithmic functions.
        This procedure is called \hl{logarithmic differentiation}.
        This process uses the properties of natural logarithm to simplify the work involved in differentiating complicated expressions containing products, quotients, and powers.
    }

    \follow\textbf{EXAMPLE 3.0.1.}
    Find the \inline{\frac{dy}{dx}} of the equation \inline{y = \frac{{(\sin{x})}^2 {(x^3 + 1)}^4}{{(x + 3)}^8}}.\
    \proving{
        \ln{(y)} &= \ln{\left[\frac{{(\sin{x})}^2 {(x^3 + 1)}^4}{{(x + 3)}^8}\right]} \\
        \ln{(y)} &= \ln{{(\sin{x})}^2} + \ln{{(x^3 + 1)}^4} \mins{} \ln{{(x + 3)}^8} \\
        \ln{(y)} &= 2 \ln{(\sin{x})} + 4 \ln{{(x^3 + 1)}} \mins{} 8 \ln{(x + 3)} \\
        \frac{1}{y} \frac{dy}{dx} &= \frac{2 \cos{x}}{\sin{x}} + \frac{4 (3x^2)}{x^3 + 1} \mins{} \frac{8}{x + 3} \\
        \frac{dy}{dx} &= y \left[\frac{2 \cos{x}}{\sin{x}} + \frac{12x^2}{x^3 + 1} \mins{} \frac{8}{x + 3}\right] \\
        \frac{dy}{dx} &= \frac{{(\sin{x})}^2 {(x^3 + 1)}^4}{{(x + 3)}^8} \left[ 2 \cot{x} + \frac{12x^2}{x^3 + 1} \mins{} \frac{8}{x + 3} \right]
    }

    \follow\textbf{NOTE.}
    The equation \inline{\ln{a} = \ln{b}} is equivalent to \inline{a = b}.

    \follow\textbf{EXAMPLE 3.0.2.}
    Solve for the equation \inline{y = x^3 \sqrt{5 \mins{} 9x}}.

    \follow\textbf{EXAMPLE 3.0.3.}
    Solve for the equation \inline{y = \frac{x^2 {(6 + 3x)}^4}{\sqrt[3]{9 \mins{} x^2}}}.

    \follow\textbf{EXAMPLE 3.0.4.}
    Solve for the equation \inline{y = x^{\sin{x}}}.
    \proving{
        \ln{(y)} &= \ln{(x^{\sin{x}})} \\
        \ln{(y)} &= \sin{x} \ln{(x)} \\
        \frac{1}{y} \frac{dy}{dx} &= (\sin{x}) (\frac{1}{x}) + (\cos{x}) (\ln{x}) \\
        \frac{1}{y} \frac{dy}{dx} &= \frac{\sin{x}}{x} + \cos{x} \ln{(x)} \\
        \frac{dy}{dx} &= y \left[\frac{\sin{x}}{x} + \cos{x} \ln{(x)}\right] \\
        \frac{dy}{dx} &= x^{\sin{x}} \left[\frac{\sin{x}}{x} + \cos{x} \ln{(x)}\right]
    }

    \follow\textbf{EXAMPLE 3.0.5.}
    Solve for the equation \inline{y = \frac{\sqrt[3]{x + 1}}{(x + 2) \sqrt{x + 3}}}.

    \follow\textbf{EXAMPLE 3.0.6.}
    Solve for the equation \inline{y = \frac{{(x \mins{} 2)}^2}{\sqrt{x^2 + 1}}}.

\end{document}
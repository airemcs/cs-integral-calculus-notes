\documentclass[12pt]{article}

\usepackage[margin=1in]{geometry}
\usepackage{xcolor, color, soul}
\usepackage{amssymb}
\usepackage{enumitem}
\usepackage{booktabs}
\usepackage{amsmath}
\usepackage{array}
\usepackage{tabularx}
\usepackage{graphicx}
\usepackage{wrapfig}
\usepackage{blindtext}

\graphicspath{ {./images/} }
\sethlcolor{yellow}

\newcommand{\class}{Derivatives of Elementary Transcendental Functions}
\newcommand{\datewritten}{Term 1, Fall '23}
\newcommand{\instructor}{Course Instructor: Noel T. Fortun}
\newcommand{\notes}{A.L. Maagma}
\newcommand{\follow}{\bigskip\noindent}
\newcommand{\spaces}{\quad\quad\quad}
\newcommand{\spacee}{\quad}
\newcommand{\point}{\,\cdot\,}

\newcommand{\mins}{-}
\newcommand{\block}[1]{\[{#1}\]}
\newcommand{\inline}[1]{\({#1}\)}
\newcommand{\proving}[1]{\begin{align*}{#1}\end{align*}}
\newcommand{\numbers}[1]{\begin{enumerate}[label*=\arabic*.]{#1}\end{enumerate}}
\newcommand{\enclose}[1]{\fbox{\parbox{\dimexpr\linewidth-2\fboxsep-2\fboxrule}{{#1}}}}

\newenvironment{conditions}
{\par\vspace{\abovedisplayskip}\noindent\begin{tabular}{>{$}l<{$} @{${}={}$} l}}
{\end{tabular}\par\vspace*{\belowdisplayskip}}

% ---------------- Start Document ----------------

\begin{document}
\pagestyle{plain}
\thispagestyle{empty}

% -------------------- Header --------------------

\noindent
\begin{tabular*}{\textwidth}{l @{\extracolsep{\fill}} r @{\extracolsep{6pt}} l}
    \textbf{\class} && \textbf{Date: \datewritten} \\
    \textbf{\instructor} && \textbf{Notes: \notes} \\
\end{tabular*}\\
\rule[2ex]{\textwidth}{2pt}

% --------------------- Body ---------------------
\noindent\section{Integration by Parts}

    \follow\textbf{EXAMPLE 1.0.1.}
    % \proving{
    %     \int{} x e^x \, dx &= 
    % }

    \follow\textbf{EXAMPLE 1.0.2.}
    \proving{
        \int{} x \ln(x) \, dx &=            & u &= \ln(x)           & dv &= x \, dx \\
        &= uv \mins{} \int{} v \, du        & du &= \frac{dx}{x}    & v &= \frac{x^2}{2} \\
        &= (\ln(x)) (\frac{x^2}{2}) \mins{} \int{} \frac{x^2}{2} \frac{dx}{x} \\
        &= \frac{x^2 \ln(x)}{2} \mins{} \frac{1}{2} \int{} x \, dx \\
        &= \frac{x^2 \ln(x)}{2} \mins{} \frac{1}{2} \point{} \frac{x^2}{2} + C \\
        &= \frac{x^2 \ln(x)}{2} \mins{} \frac{x^2}{4} + C
    }

    \follow\textbf{EXAMPLE 1.0.3.}
    \proving{
        \int{} x \sec(x) \tan(x) \, dx &= x \sec(x) \mins{} \int{} \sec(x) \, dx    & u &= x        & dv &= \sec(x) \tan(x) \, dx \\
        &= x \sec(x) \mins{} \ln{|\sec(x) + \tan(x)|} + C                           & du &= dx      & v &= \sec(x) \\
    }

    \follow\textbf{EXAMPLE 1.0.3.}
    \proving{
        \int{} \ln(x) \, dx &= x \ln(x) \mins{} \int{} x \frac{dx}{x}   & u &= \ln(x)           & dv &= dx \\
        &= x \ln(x) \mins{} x + C                                       & du &= \frac{dx}{x}    & v &= x
    }

    \follow\textbf{EXAMPLE 1.0.4.}
    \proving{
        \int{} x \cos(x) \, dx &= uv \mins{} \int{} v du    & u &= x        & dv &= \cos(x) \, dx \\
        &= x \sin(x) \mins{} \int{} \sin(x) dx              & du &= dx      & v &= \sin(x) \\
        &= x \sin(x) + \cos(x) + C
    }

    \follow\textbf{EXAMPLE 1.0.5.}
    \proving{
        \int{} \sin(x) \ln(\cos(x)) \, dx &= \mins{} \cos(x) \ln(\cos(x)) + \cos(x) + C       & u &= \ln(\cos(x))          & dv &= x
    }

    \follow\textbf{EXAMPLE 1.0.6.}
    \proving{
        \int_{1}^{2} \frac{\ln(x)}{x^2} \, dx &= \int_{1}^{2} \ln(x) \, x^{\mins{} 2} \, dx                                 & u &= \ln(x)           & dv &= x^{\mins{} 2} \, dx \\
        &= {\left[(\ln(x)) (\mins{} x^{\mins{} 1})\right]}_{1}^{2} \mins{} \int_{1}^{2} \mins{} x^{\mins{} 1} \frac{dx}{x}  & du &= \frac{dx}{x}    & v &= \mins{} x^{\mins{} 1} \\
        &= \mins{} \left[\frac{\ln(x)}{x}\right]_{1}^{2} + \int_{1}^{2} x^{-2} \, dx \\
        &= {\left[\mins{} \frac{\ln(x)}{x} + \frac{x^{-1}}{-1}\right]}^{2}_{1} \\
        &= {\left[\frac{\mins{} \ln(x) \mins{} 1}{x}\right]}^{2}_{1} \\
        &= \frac{\mins{} \ln(2) \mins{} 1}{2} \mins{} \frac{-1}{1} \\
        &= \frac{\mins{} \ln(2) \mins{} 1}{2} + 1 \\
        &= \frac{\mins{} \ln(2) + 1}{2}
    }

    \follow\textbf{EXAMPLE 1.0.7.}
    \proving{
        \int{} x^2 e^x \, dx &= x^2 e^x \mins{} \int{} e^x (2x dx)      &= u &= x^2             & dv &= e^x \, dx \\
        &= x^2 e^x \mins{} 2 \int{} x e^x \, dx                         & du &= 2x \, dx        & v &= e^x \\
    }

    \follow\textbf{EXAMPLE 1.0.8.}
    \proving{
        \int{} e^x \sin(x) \, dx &= e^x \sin(x) \mins{} \int{} e^x \cos(x) \, dx    & u &= \sin(x)      & dv &= e^x \, dx \\
        &= & du_1 &= \cos(x) \, dx      & v_1 &= e^x
    }

    \newpage\follow\textbf{EXAMPLE 1.0.9.}
    \proving{
        \int{} e^x \cos(x) \, dx &= e^x \cos(x) \mins{} \int{} e^x (\mins{} \sin(x) \, dx)      & u_1 &= \cos(x)                        & dv_1 &= e^x \, dx \\
                                                                                                && du_1 &= \mins{} \sin(x) \, dx        & v_1 &= e^x \\
        &= e^x \cos(x) + \int{} e^x \sin(x) \, dx                                               & u_2 &= \sin(x)                        & dv_2 &= e^x \, dx \\
                                                                                                && du_2 &= \cos(x)                      & v_2 &= e^x \\
        &= e^x \cos(x) + (e^x \sin(x) \mins{} \int{} e^x (\cos(x))) \\
        e^x \cos(x) \, dx &= e^x \cos(x) + e^x \sin(x) \mins{} \int{} e^x \cos(x) \\
        \frac{2 \int{} e^x \cos(x) \, dx}{2} &= \frac{e^x \cos(x) + e^x \sin(x)}{2} + C \\
        \int{} e^x \cos(x) \, dx &= \frac{e^x \cos(x) + e^x \sin(x)}{2} + C
    }

\end{document}
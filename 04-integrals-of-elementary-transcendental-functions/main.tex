\documentclass[12pt]{article}

\usepackage[margin=1in]{geometry}
\usepackage{xcolor, color, soul}
\usepackage{amssymb}
\usepackage{enumitem}
\usepackage{booktabs}
\usepackage{amsmath}
\usepackage{array}
\usepackage{tabularx}
\usepackage{graphicx}
\usepackage{wrapfig}
\usepackage{blindtext}

\graphicspath{ {./images/} }
\sethlcolor{yellow}

\newcommand{\class}{Integrals of Elementary Transcendental Functions}
\newcommand{\datewritten}{Term 1, Fall '23}
\newcommand{\instructor}{Course Instructor: Noel T. Fortun}
\newcommand{\notes}{A.L. Maagma}
\newcommand{\follow}{\bigskip\noindent}
\newcommand{\spaces}{\quad\quad\quad}
\newcommand{\spacee}{\quad}
\newcommand{\point}{\,\cdot\,}

\newcommand{\mins}{-}
\newcommand{\block}[1]{\[{#1}\]}
\newcommand{\inline}[1]{\({#1}\)}
\newcommand{\proving}[1]{\begin{align*}{#1}\end{align*}}
\newcommand{\numbers}[1]{\begin{enumerate}[label*=\arabic*.]{#1}\end{enumerate}}
\newcommand{\enclose}[1]{\fbox{\parbox{\dimexpr\linewidth-2\fboxsep-2\fboxrule}{{#1}}}}

\newenvironment{conditions}
{\par\vspace{\abovedisplayskip}\noindent\begin{tabular}{>{$}l<{$} @{${}={}$} l}}
{\end{tabular}\par\vspace*{\belowdisplayskip}}

% ---------------- Start Document ----------------

\begin{document}
\pagestyle{plain}
\thispagestyle{empty}

% -------------------- Header --------------------

\noindent
\begin{tabular*}{\textwidth}{l @{\extracolsep{\fill}} r @{\extracolsep{6pt}} l}
    \textbf{\class} && \textbf{Date: \datewritten} \\
    \textbf{\instructor} && \textbf{Notes: \notes} \\
\end{tabular*}\\
\rule[2ex]{\textwidth}{2pt}

% --------------------- Body ---------------------

\section{Integrals Yielding the Natural Logarithmic Function}

    \enclose{
        \textbf{DEFINITION.}
        Since the natural logarithm is undefined for negative numbers, you will often encounter expressions of the form \inline{\ln{|u|}}.
        Hence, for derivatives involving absolute values, if \textit{u} is a differentiable function of \textit{x} such that \inline{u \neq{} 0}, then \block{\frac{d}{dx} [\ln{|u|}] = \frac{u'}{u}}
    }

    \follow\textbf{EXAMPLE 1.0.0.} Evaluate the equation \inline{\int{} \frac{1}{u} du}.
    \proving{
        &= \ln{|u|} + C
    }

    \follow\enclose{
        \textbf{DEFINITION.}
        For any rational number \inline{n \neq{} \mins{} 1}, \block{\int{} u^n \, du = \frac{u^{n + 1}}{n + 1} + C}
        However, for any rational number \inline{n = \mins{} 1}, \block{\int{} u^n \, du = \ln{|u|} + C}
    }

    \follow\textbf{EXAMPLE 1.0.1.} Evaluate the equation \inline{\int{} \frac{2}{x} \, dx}.

    \follow\textbf{EXAMPLE 1.0.2.} Evaluate the equation \inline{\int{} \frac{dx}{4x \mins{} 1}}.

    \follow\textbf{EXAMPLE 1.0.3.} Find the area of the region on the \textit{x}-axis and the line \inline{x = 3}, bounded by the graph of \block{y = \frac{x}{x^2 + 1}}

    \follow\textbf{EXAMPLE 1.0.4.} Evaluate the equation \inline{\int{} \frac{x + 1}{x^2 + 2x} \, dx}.

    \follow\textbf{EXAMPLE 1.0.5.} Evaluate the equation \inline{\int{} \frac{\sec^2{x}}{\tan{x}} \, dx}.

    \follow\textbf{EXAMPLE 1.0.6.} Evaluate the equation \inline{\int{} \frac{\ln{x}}{x} \, dx}.

    \follow\textbf{NOTE.}
    The integrals to which this formula or rules can be applied may appear in disguised form.
    For instance, when a \hl{rational function has a numerator of degree greater than or equal to that of the denominator}, division may reveal a form to which you can apply the rule.

    \follow\textbf{EXAMPLE 1.0.7.} Evaluate the equation \inline{\int{} \frac{x^2 + x + 1}{x^2 + 1} \, dx}.

    \follow\textbf{EXAMPLE 1.0.8.} Evaluate the equation \inline{\int_{0}^{2} \frac{x^2 + 2}{x + 1} \, dx}.

\section{Integrals Involving Logarithmic Functions}

    \follow\textbf{EXAMPLE 2.0.1.} Evaluate the equation \inline{\int{} \frac{\log_{10} x}{x}}.

    \follow\textbf{EXAMPLE 2.0.2.} Evaluate the equation \inline{\int{} \frac{{(\log_{3} x)}^2}{x}}.

\section{Integral of Trigonometric Functions}

    \follow\textbf{EXAMPLE 3.0.1.} Evaluate the equation \inline{\int{} \tan{x} \, dx}.
    \proving{
        &= \mins{} \int{} \frac{\mins{} \sin{x}}{\cos{x}} \, dx     & u &= \cos{x}                  \\
        &= \mins{} \int{} \frac{du}{u}                              & du &= \mins{} \sin{x} \, dx   \\
        &= \mins{} \ln{|u|} + C \\
        &= \mins{} \ln{|\cos{x}|} + C \\
        &= \ln{|{(\cos{x})}^{\mins{} 1}|} + C \\
        &= \ln{|\sec{x}|} + C
    }

    \follow\textbf{EXAMPLE 3.0.2.} Evaluate the equation \inline{\int{} \cot{x} \, dx}.
    \proving{
        &= \int{} \frac{\cos{x}}{\sin{x}} \, dx     & u &= \sin{x} \\
        &= \int{} \frac{du}{u}                      & du &= \cos{x} \, dx \\
        &= \ln{|u|} + C \\
        &= \ln{|\sin{x}|} + C
    }

    

\end{document}
\documentclass[12pt]{article}

\usepackage[margin=1in]{geometry}
\usepackage{xcolor, color, soul}
\usepackage{amssymb}
\usepackage{enumitem}
\usepackage{booktabs}
\usepackage{amsmath}
\usepackage{array}
\usepackage{tabularx}
\usepackage{graphicx}
\usepackage{wrapfig}
\usepackage{blindtext}

\graphicspath{ {./images/} }
\sethlcolor{yellow}

\newcommand{\class}{Application of Definite Integrals}
\newcommand{\datewritten}{Term 1, Fall '23}
\newcommand{\instructor}{Course Instructor: Noel T. Fortun}
\newcommand{\notes}{A.L. Maagma}
\newcommand{\follow}{\bigskip\noindent}
\newcommand{\spaces}{\quad\quad\quad}
\newcommand{\spacee}{\quad}
\newcommand{\point}{\,\cdot\,}

\newcommand{\mins}{-}
\newcommand{\block}[1]{\[{#1}\]}
\newcommand{\inline}[1]{\({#1}\)}
\newcommand{\proving}[1]{\begin{align*}{#1}\end{align*}}
\newcommand{\numbers}[1]{\begin{enumerate}[label*=\arabic*.]{#1}\end{enumerate}}
\newcommand{\enclose}[1]{\fbox{\parbox{\dimexpr\linewidth-2\fboxsep-2\fboxrule}{{#1}}}}

\newenvironment{conditions}
{\par\vspace{\abovedisplayskip}\noindent\begin{tabular}{>{$}l<{$} @{${}={}$} l}}
{\end{tabular}\par\vspace*{\belowdisplayskip}}

% ---------------- Start Document ----------------

\begin{document}
\pagestyle{plain}
\thispagestyle{empty}

% -------------------- Header --------------------

\noindent
\begin{tabular*}{\textwidth}{l @{\extracolsep{\fill}} r @{\extracolsep{6pt}} l}
    \textbf{\class} && \textbf{Date: \datewritten} \\
    \textbf{\instructor} && \textbf{Notes: \notes} \\
\end{tabular*}\\
\rule[2ex]{\textwidth}{2pt}

% --------------------- Body ---------------------

\noindent\section{Area of a Plane Region}

    \enclose{
        \textbf{DEFINITION.}
        It is using integrals to find areas of regions that lie between the graphs of two functions.

        \bigskip\textbf{RECALL.}
        The \hl{definite integral} generalizes the concept of the \hl{area under a curve}.
        If \textit{f} is \hl{continuous and non-negative} on \inline{[a, b]}, then the \hl{area under the graph of \textit{f}} from \inline{x = a} to \inline{x = b} is given by the integral of \textit{f} from \inline{x = a} to \inline{x = b}.
        \block{\text{Area of S =} \int_{a}^{b} f (x) \, dx}
    }
    
    \follow\textbf{EXAMPLE 1.0.1.}
    Find the area of the region bounded by the parabola \inline{y = 10 \mins{} x^2}, \textit{x}-axis, \textit{y}-axis, and \inline{x = 2}.

    \subsection{Area of Plane Region Between 2 Curves}

        \enclose{
            \textbf{DEFINITION.}
            If \textit{f} and \textit{g} continuous functions on \inline{[a, b]} and \inline{f (x) \geq{} g (x)} for all \inline{x \in{} [a, b]}, then the \hl{area \textit{A} of the region} bounded by the curves \inline{y = f (x)}, \inline{y = g (x)}, and the lines \inline{x = a} and \inline{x = b} is given by the definite integral \block{A = \int_{a}^{b} [f (x) \mins{} g (x)] \, dx}
        
            \follow\textbf{NOTE.}
            The formula provided on the left equates to an approximate, while the formula on the right equates to the exact area of an area.
            \proving{
                & \sum_{i = 1}^{n} f (x_{i}^{*}) \triangle{x}   &&= \lim_{n \rightarrow{} \infty{}} f (x_{i}^{*}) \triangle{x} \\
                &                                               &&= \int_{a}^{b} f (x) \, dx
            }
        }

        \newpage\follow\textbf{EXAMPLE 1.1.1}
        Find the area bounded above by \inline{y = 2x + 5} and bounded below by \inline{y = x^3} on [0, 2].
        \proving{
            A   &= \int_{a}^{b} [f (x) \mins{} g (x)] \, dx \\
                &= \int_{a}^{b} 2x + 5 \mins{} x^3 \, dx \\
                &= {\left[ x^2 + 5x \mins{} \frac{x^4}{4} \right]}_{0}^{2} \\
                &= 4 + 10 \mins{} 4 \mins{} 0 \\
                &= 10
        }
        
        \follow\textbf{EXAMPLE 1.1.2.}
        Find the area of the region bounded above by the parabola \inline{y = 9 \mins{} x^2} and the line \inline{y = 2x + 1}.
        \proving{
            &= \int_{\mins{} 4}^{2} (9 \mins{} x^2 \mins{} 2x \mins{} 1) \, dx \\
            &= {\left[ 9x \mins{} \frac{x^3}{3} \mins{} x^2 \mins{} x \right]}_{\mins{} 4}^{2} \\
            &= 36
        }

        \follow\textbf{EXAMPLE 1.1.3.}
        Find the area of the region bounded by the parabolas \inline{y = x^2} and \inline{y = \mins{} x^2 + 4x}.
        \proving{
            x^2 &= \mins{} x^2 + 4x     & 1 &= \mins{} 1 + 4    & &= \int_{0}^{2} (\mins{} x^2 + 4x \mins{} x^2) \, dx \\
            2x^2 &= 4x                  & 1 &= 3                & &= \int_{0}^{2} (\mins{} 2x^2 + 4x) \, dx \\
            x &= 2, 0                   &&                      & &= {\left[ \mins{} \frac{2x^3}{3} + 2x^2 \right]}_{0}^{2} \\
            &                           &&                      & &= \mins{} \frac{16}{3} + 8 \\
            &                           &&                      & &= \frac{8}{3}
        }

        \follow\textbf{NOTE.}
        When selecting for a value of substituting, the values of \textit{x} must be between only those calculated.

        \newpage\textbf{EXAMPLE 1.1.3.}
        Find the area bounded by \inline{y = x^3} and \inline{y = x}.
        \proving{
            x^3 &= x            \\
            x^2 &= 1            \\
            x &= \pm{} 1, 0
        }

\end{document}
\documentclass[12pt]{article}

\usepackage[margin=1in]{geometry}
\usepackage{xcolor, color, soul}
\usepackage{amssymb}
\usepackage{enumitem}
\usepackage{booktabs}
\usepackage{amsmath}
\usepackage{array}
\usepackage{tabularx}
\usepackage{graphicx}
\usepackage{wrapfig}
\usepackage{blindtext}

\graphicspath{ {./images/} }
\sethlcolor{yellow}

\newcommand{\class}{Application of Definite Integrals}
\newcommand{\datewritten}{Term 1, Fall '23}
\newcommand{\instructor}{Course Instructor: Noel T. Fortun}
\newcommand{\notes}{A.L. Maagma}
\newcommand{\follow}{\bigskip\noindent}
\newcommand{\spaces}{\quad\quad\quad}
\newcommand{\spacee}{\quad}
\newcommand{\point}{\,\cdot\,}

\newcommand{\mins}{-}
\newcommand{\block}[1]{\[{#1}\]}
\newcommand{\inline}[1]{\({#1}\)}
\newcommand{\proving}[1]{\begin{align*}{#1}\end{align*}}
\newcommand{\numbers}[1]{\begin{enumerate}[label*=\arabic*.]{#1}\end{enumerate}}
\newcommand{\enclose}[1]{\fbox{\parbox{\dimexpr\linewidth-2\fboxsep-2\fboxrule}{{#1}}}}

\newenvironment{conditions}
{\par\vspace{\abovedisplayskip}\noindent\begin{tabular}{>{$}l<{$} @{${}={}$} l}}
{\end{tabular}\par\vspace*{\belowdisplayskip}}

% ---------------- Start Document ----------------

\begin{document}
\pagestyle{plain}
\thispagestyle{empty}

% -------------------- Header --------------------

\noindent
\begin{tabular*}{\textwidth}{l @{\extracolsep{\fill}} r @{\extracolsep{6pt}} l}
    \textbf{\class} && \textbf{Date: \datewritten} \\
    \textbf{\instructor} && \textbf{Notes: \notes} \\
\end{tabular*}\\
\rule[2ex]{\textwidth}{2pt}

% --------------------- Body ---------------------

\noindent\section{Area of a Plane Region}

    \enclose{
        \textbf{DEFINITION.}
        It is using integrals to find areas of regions that lie between the graphs of two functions.

        \bigskip\textbf{RECALL.}
        The \hl{definite integral} generalizes the concept of the \hl{area under a curve}.
        If \textit{f} is \hl{continuous and non-negative} on \inline{[a, b]}, then the \hl{area under the graph of \textit{f}} from \inline{x = a} to \inline{x = b} is given by the integral of \textit{f} from \inline{x = a} to \inline{x = b}.
        \block{\text{Area of S =} \int_{a}^{b} f (x) \, dx}
    }
    
    \follow\textbf{EXAMPLE 1.0.1.}
    Find the area of the region bounded by the parabola \inline{y = 10 \mins{} x^2}, \textit{x}-axis, \textit{y}-axis, and \inline{x = 2}.

    \subsection{Area of Plane Region Between 2 Curves}

        \enclose{
            \textbf{DEFINITION.}
            If \textit{f} and \textit{g} continuous functions on \inline{[a, b]} and \inline{f (x) \geq{} g (x)} for all \inline{x \in{} [a, b]}, then the \hl{area \textit{A} of the region} bounded by the curves \inline{y = f (x)}, \inline{y = g (x)}, and the lines \inline{x = a} and \inline{x = b} is given by the definite integral \block{A = \int_{a}^{b} [f (x) \mins{} g (x)] \, dx}
        
            \follow\textbf{NOTE.}
            The formula provided on the left equates to an approximate, while the formula on the right equates to the exact area of an area.
            \proving{
                & \sum_{i = 1}^{n} f (x_{i}^{*}) \triangle{x}   &&= \lim_{n \rightarrow{} \infty{}} f (x_{i}^{*}) \triangle{x} \\
                &                                               &&= \int_{a}^{b} f (x) \, dx
            }
        }

        \newpage\follow\textbf{EXAMPLE 1.1.1}
        Find the area bounded above by \inline{y = 2x + 5} and bounded below by \inline{y = x^3} on [0, 2].
        \proving{
            A   &= \int_{a}^{b} [f (x) \mins{} g (x)] \, dx \\
                &= \int_{a}^{b} 2x + 5 \mins{} x^3 \, dx \\
                &= {\left[ x^2 + 5x \mins{} \frac{x^4}{4} \right]}_{0}^{2} \\
                &= 4 + 10 \mins{} 4 \mins{} 0 \\
                &= 10
        }
        
        \follow\textbf{EXAMPLE 1.1.2.}
        Find the area of the region bounded above by the parabola \inline{y = 9 \mins{} x^2} and the line \inline{y = 2x + 1}.
        \proving{
            &= \int_{\mins{} 4}^{2} (9 \mins{} x^2 \mins{} 2x \mins{} 1) \, dx \\
            &= {\left[ 9x \mins{} \frac{x^3}{3} \mins{} x^2 \mins{} x \right]}_{\mins{} 4}^{2} \\
            &= 36
        }

        \follow\textbf{EXAMPLE 1.1.3.}
        Find the area of the region bounded by the parabolas \inline{y = x^2} and \inline{y = \mins{} x^2 + 4x}.
        \proving{
            x^2 &= \mins{} x^2 + 4x     & 1 &= \mins{} 1 + 4    & &= \int_{0}^{2} (\mins{} x^2 + 4x \mins{} x^2) \, dx \\
            2x^2 &= 4x                  & 1 &= 3                & &= \int_{0}^{2} (\mins{} 2x^2 + 4x) \, dx \\
            x &= 2, 0                   &&                      & &= {\left[ \mins{} \frac{2x^3}{3} + 2x^2 \right]}_{0}^{2} \\
            &                           &&                      & &= \mins{} \frac{16}{3} + 8 \\
            &                           &&                      & &= \frac{8}{3}
        }

        \follow\textbf{NOTE.}
        When selecting for a value of substituting, the values of \textit{x} must be between only those calculated.

        \newpage\follow\textbf{EXAMPLE 1.1.3.}
        Find the area bounded by \inline{y = x^3} and \inline{y = x}.
        \proving{
            x^3 &= x            \\
            x^2 &= 1            \\
            x &= \pm{} 1, 0
        }

        \follow\textbf{EXAMPLE 1.1.4.}
        Find the area bounded by \inline{x = y^2} and \inline{y = x \mins{} 2}.
        \proving{
            x &= y^2            & y &= x \mins{} 2          & &= \int_{c}^{d} \left[ f (y) \mins{} g (y) \right] \, dy \\
            g (y) &= y^2        & y + 2 &= x                & &= \int_{-1}^{2} \left(y + 2 \mins{} y^2 \right) \, dy \\
            &                   & y + 2 &= f (y)            & &= {\left[\frac{y^2}{2} + 2y \mins{} \frac{y^3}{3}\right]}_{-1}^{2} \\
            &                   &&                          & &= \frac{9}{2}
        }

        \follow\textbf{NOTE.}
        There are some regions wherein they are best treated by regarding \textit{x} as a function of \textit{y}.
        
        \follow\textbf{EXAMPLE 1.1.5.}
        Find the area bounded by \inline{x = 2y^2} and \inline{x = 4 + y^2}. You must decide whether to integrate with respect to \textit{x} or \textit{y}.
        \proving{
            2y^2 &= 4 + y^2         & 0 &= 4 + 0            & &= \int_{c}^{d} \left[f (y) \mins{} g (y) \right] \, dy \\
            y^2 &= 4                & 0 &= 4                & &= \int_{-2}^{2} \left[4 + y^2 \mins{} 2y^2 \right] \, dy \\
            y &= \pm{} 2            &&                      & &= \int_{-2}^{2} \left(4 \mins{} y^2 \right) \, dy \\
            &                       &&                      & &= \frac{32}{3}
        }

        \follow\textbf{EXAMPLE 1.1.6.}
        Find the area bounded by \inline{x = y^3} and \inline{x = \mins{} 3y^2 + 4}.
        \proving{
            y^3 &= \mins{} 3y^2 + 4                         & &= \int_{-2}^{1} \left[ \mins{} 3y^2 + 4 \mins{} y^3 \right] \, dy \\
            y^3 + 3y^2 \mins{} 4 &= 0 \\
            (y \mins{} 1) (y + 2) (y + 2) &= 0 \\
            y &= 1, \mins{} 2
        }

        \newpage\follow\textbf{EXAMPLE 1.1.7.}
        Find the area bound by \inline{y = 5x \mins{} x^2} and \inline{y = x}.
        \proving{
            5x \mins{} x^2 &= x                             & &= \int_{0}^{4} (5x \mins{} x^2 \mins{} x) \, dx \\
            x^2 + x \mins{} 5x &= 0                         & &= \int_{0}^{4} (4x \mins{} x^2) \, dx \\
            x^2 \mins{} 4x &= 0                             & &= {\left[ 2x^2 \mins{} \frac{x^3}{3} \right]}_{0}^{4} \\
            x (x \mins{} 4) &= 0                            & &= 2 (4)^2 \mins{} \frac{(4)^3}{3} \\
            x &= 0, 4                                       & &= 32 \mins{} \frac{64}{3} \\
            &                                               & &= \frac{32}{3}
        }

        \follow\textbf{EXAMPLE 1.1.8.}
        Find the area bound by \inline{x = y^2 \mins{} 4y} and \inline{x = 2y \mins{} y^2}.
        \proving{
            y^2 \mins{} 4y &= 2y \mins{} y^2                & &= \int_{0}^{3} (2y \mins{} y^2 \mins{} y^2 + 4y) \, dy \\
            2y^2 \mins{} 6y &= 0                            & &= \int_{0}^{3} (6y \mins{} 2y^2) \, dy \\
            y^2 \mins{} 3y &= 0                             & &= {\left[ 3y^2 \mins{} \frac{2y^3}{3} \right]}_{0}^{3} \\
            y (y \mins{} 3) &= 0                            & &= 3 (3)^2 \mins{} \frac{2 (3)^3}{3} \\
            y &= 0, 3                                       & &= 9
        }

\noindent\section{Volumes of Solids of Revolution}

    \enclose{
        \textbf{DEFINITION}
        It is using integration to find out the volume of a \hl{solid of revolution}.
        We have an intuitive idea of what volume means.
        In calculus, we make this idea precise to give an exact definition of volume.
    }

    \subsection{Solid of Revolution}

        \enclose{
            \textbf{DEFINITION.}
            It is a solid obtained by revolving a plane region about a fixed line called the \hl{axis of revolution}.
        }
    
    \subsection{Disk Method}

        \enclose{
            \textbf{DEFINITION.}
            Let \textit{f} be continuous with \inline{f (x) \geq{} 0} on the interval \inline{[a, b]}.
            If the region \textit{R} bounded by the graph of \textit{f}, the \textit{x}-axis, and the lines \inline{x = a} and \inline{x = b} is \hl{revolved around the \textit{x}-axis}, the volume of the resulting solid is \block{V = \int_{a}^{b} \pi{} f (x)^2 \, dx = \pi{} \int_{a}^{b} f (x)^2 \, dx}
            Additionally, the set-up goes:
            \begin{itemize}
                \setlength\itemsep{0em}
                \item[-] For a solid \textit{S} that isn't a cylinder, we first ``cut'' \textit{S} into pieces and approximate each piece by a thin cylinder or disk.
                \item[-] We estimate the volume of \textit{S} by adding the volumes of the disk.
                \item[-] We arrive at the exact volume of \textit{S} through a \hl{limiting process} in which the number of pieces become large.
            \end{itemize}
        }

        \follow\textbf{EXAMPLE 2.2.1.}
        Given that the axis of revolution is x-axis, find the volume of the solid obtained by rotating about the \textit{x}-axis of the region under the curve \inline{y = \sqrt{x}} from 0 to 1.
        Illustrate the definition of volume by sketching a typical approximating cylinder.
        \proving{
            V &= \int_{a}^{b} A (x) \, dx \\
            &= \int_{0}^{1} \pi{} (\sqrt{x})^2 \, dx \\
            &= \pi{} \int_{0}^{1} x \, dx \\
            &= \pi{} {\left[ \frac{x^2}{2} \right]}_{0}^{1} \\
            &= \frac{\pi{}}{2}
        }

        \follow\textbf{EXAMPLE 2.2.2.}
        Given that it is rotated around the \textit{x}-axis, write the integral that would be used to find the volume of the region bounded by \inline{x = -1}, \inline{x = 2}, \inline{y = 0}, and \inline{y = \frac{1}{2} x^2 + 2}.
        \proving{
            V &= \int_{a}^{b} \pi{} f (x)^2 \, dx \\
            &= \pi{} \int_{-1}^{2} {\left(\frac{1}{2} x^2 + 2\right)}^2 \, dx \\
            &= \pi{} \int_{-1}^{2} {\left(\frac{1}{2} x^2 + 2\right)}^2 \, dx
        }

        \newpage\follow\textbf{EXAMPLE 2.2.3.}
        Given that the axis of revolution is not a coordinate axis, find the volume of the solid when the region bounded by \inline{f (x) = 2 \mins{} x^2} and \inline{g (x) = 1} is revolved about the line \inline{y = 1}.
        \proving{
            V &= \pi{} \int_{-1}^{1} {(1 \mins{} x^2)}^2 \, dx \\
            &= \pi{} \int_{-1}^{1} (1 \mins{} 2x^2 + x^4) \, dx \\
            &= \pi{} {\left[ x \mins{} \frac{2x^3}{3} + \frac{x^5}{5} \right]}_{-1}^{1} \\
            &= \frac{16 \pi{}}{15}
        }

        \follow\textbf{EXAMPLE 2.2.4.}
        Given that the axis of revolution is \textit{y}-axis, find the volume of the solid obtained by rotating the region bounded by \inline{y = x^3}, \inline{y = 8}, and \inline{x = 0} about the \textit{y}-axis.
        \proving{
            V &= \int_{y_1}^{y_2} \pi{} f (y)^2 \, dy \\
            &= \pi{} \int_{0}^{8} {(\sqrt[3]{y})}^2 \, dy \\
            &= \pi{} \int_{0}^{8} y^{2/3} \, dy \\
            &= \pi{} {\left[ \frac{y^{5/3}}{5/3} \right]}_{0}^{8} \\
            &= \frac{3 \pi{}}{5} {\left[ y^{5/3} \right]}_{0}^{8} \\
            &= \frac{96 \pi{}}{5}
        }

        \follow\textbf{EXAMPLE 2.2.5.}
        Write the integral that would be used to find the volume of the solid obtained by revolving the region bounded by \inline{x = y^2 \mins{} 4}, \inline{x = 0}, and \inline{y = 0} about the \textit{y}-axis.
        \proving{
            r &= 0 \mins{} (y^2 \mins{} 4)          & A (y) &= \pi{} {(y^2 \mins{} 4)}^2 \\
            r^2 &= {(y^2 \mins{} 4)}^2              & V &= \int_{y_1}^{y_2} A (y) \, dy \\
            &                                       & &= \int_{0}^{2} \pi{} {(y^2 \mins{} 4)}^2 \, dy
        }

        \newpage\follow\textbf{EXAMPLE 2.2.6.}
        Given that the axis of revolution is not the \textit{y}-axis, find the volume of the solid when the region bounded by \inline{x = y^2} and \inline{x = 1} is revolved about the line \inline{x = 1}.
        \proving{
            V &= \int_{-1}^{1} \pi{} {(1 \mins{} y^2)}^2 \, dy \\
            &= \pi{} \int_{-1}^{1} (1 \mins{} 2y^2 + y^4) \, dy \\
            &= \pi{} {\left[ y \mins{} \frac{2y^3}{3} + \frac{y^5}{5} \right]}_{-1}^{1} \\
            &= \frac{16 \pi{}}{15}
        }

        \follow\enclose{
            \textbf{DEFINITION.}
            To find the volume of a solid of revolution with the disk method, use one of the formulas below.
            \follow\begin{center}\begin{tabular}{ccc}
                Horizontal Axis of Revolution && Vertical Axis of Revolution \\
                \inline{V = \pi{} \int_{a}^{b} {[r (x)]}^2 \, dx} && \inline{V = \pi{} \int_{c}^{d} {[r (y)]}^2 \, dy} \\\\
            \end{tabular}\end{center}
            
            \textbf{NOTE.}
            You can determine the variable of integration by placing a representative rectangle in the plane region \hl{perpendicular} to the axis of revolution.
            When the width of the rectangle is \inline{\triangle{x}} integrate with respect to \textit{x}, and when the width of the rectangle is \inline{\triangle{y}}, integrate with respect to \textit{y}.
        }
    
    \subsection{}

\end{document}
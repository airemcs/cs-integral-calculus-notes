\documentclass[12pt]{article}
\usepackage[margin=1in]{geometry}
\usepackage{xcolor, color, soul}
\usepackage{amssymb}
\usepackage{enumitem}

\sethlcolor{yellow}

\newcommand{\class}{Integral Calculus for Computer Science Students}
\newcommand{\datewritten}{Term 1, Fall '23}
\newcommand{\instructor}{Course Instructor: Noel T. Fortun}
\newcommand{\notes}{A.L. Maagma}
\newcommand{\follow}{\bigskip\noindent}
\newcommand{\spaces}{\quad\quad\quad}

% ---------------- Start Document ----------------

\begin{document}
\pagestyle{plain}
\thispagestyle{empty}

% -------------------- Header --------------------

\noindent
\begin{tabular*}{\textwidth}{l @{\extracolsep{\fill}} r @{\extracolsep{6pt}} l}
    \textbf{\class} && \textbf{Date: \datewritten} \\
    \textbf{\instructor} && \textbf{Notes: \notes} \\
\end{tabular*}\\
\rule[2ex]{\textwidth}{2pt}

% ------------- Topic: Differentials -------------

\section{The Differentials}

    \textbf{DEFINITION.}
    These provide us with a way of \hl{estimating the amount a function changes as a result of a small change in input values}. 
    The equation \[\triangle{y} \approx dy\] can and will only be considered if \(\triangle{x}\) is ``close enough''.
    The approximation of the equation becomes better as \(\triangle{x}\) becomes smaller.

    \follow\textbf{NOTE.}
    We equate \(\triangle{x} = dx\).

\subsection{The Differential of the Independent Variable}

    \textbf{DEFINITION.}
    If the function \textit{f} is defined by the equation \(y = f(x)\), then the \hl{differential of y}, denoted by \(dy\), is given by \[dy = f'(x)dx \longrightarrow f'(x) = \frac{dy}{dx}\] where \textit{x} is any number in the domain of \textit{f'} and \(\triangle{x}\) is an arbitrary increment of \textit{x}.

    \follow\textbf{EXAMPLE 1.1.}
    Find \(dy\) for \(y = {(x^3 + 5x - 1)}^{2023}\).
    \[f'(x) = 2023{(x^3 + 5x - 1)}^{2022}(3x^2 + 5)\]
    \[\therefore dy = 2023(x^3 + 5x - 1)(3x^2 + 5)dx\]

    \follow\textbf{EXAMPLE 1.2.}
    Find the differential \(dy\) of the function \(y = 4x^2 + x + 3\).
    \[(8x + 1)dx\]

    \follow\textbf{EXAMPLE 1.3.}
    Find the differential \(dy\) of the function \(y = \cos(x)\).
    \[dy = -\sin(x)dx\]

    \newpage\follow\textbf{EXAMPLE 1.4.}
    Compare the values of \(\triangle{y}\) and \(dy\) if \(y = f(x) = x^3 + x^2 - 2x + 1\) and \(x\) changes (a) from 2 to 2.05 and (b) from 2 to 2.01.
    \[x = 2 \spaces x + \triangle{x} = 2.05\]
    \[\Rightarrow \triangle{x} = 0.05 = dx\]

    \[\triangle{y} = f(x + \triangle{x}) - f(x)\]
    \[= f(2.05) - f(2)\]
    \[= [{(2.05)}^3 + {(2.05)}^2 + 2(2.05) + 1] - [2^3 + 2^2 - 2.2 + 1]\]
    \[\triangle{y} = 0.7176\]

    \[dy = f'(x)dx\]
    \[= (3x^2 + 2x - 2)dx\]
    \[= (3.2^2 + 2.2 - 2)(0.05)\]
    \[dy = 0.7\]

    \follow\textbf{NOTE.}
    The final equation utilized our solution at \(x = 2, \triangle{x} = dx = 0.05\).
    Observe that the approximation of \(\triangle{y} \approx dy\) becomes better as \(\triangle{x}\) becomes smaller.

\newpage\subsection*{Practice Exercises}

    \begin{enumerate}
        
        \item Find \(dy\) and \(\triangle{y}\) for the given values of \(x\) and \(\triangle{x}\).
        \begin{enumerate}[label*=\arabic*.]
            \item \(y = x^2\), \(x = 2\), and \(\triangle{x} = 0.5\)
            \item \(y = x^3\), \(x = 2\), and \(\triangle{x} = 0.5\)
            \item \(y = \sqrt[3]{x}\), \(x = 8\), and \(\triangle{x} = 1\)
            \item \(y = \sqrt[]{x}\), \(x = 4\), and \(\triangle{x} = 1\)
        \end{enumerate}

        \item Find (a) \(\triangle{y}\); (b) \(dy\); (c) \(\triangle{y} - dy\).
        \begin{enumerate}[label*=\arabic*.]
            \item \(y = x^2 - 3x\), \(x = 2\), and \(\triangle{x} = 0.03\)
            \item \(y = x^2 - 3x\), \(x = -1\), and \(\triangle{x} = 0.02\)
            \item \(y = \frac{1}{x}\), \(x = -2\), and \(\triangle{x} = -0.1\)
            \item \(y = \frac{1}{x}\), \(x = 3\), and \(\triangle{x} = -0.2\)
            \item \(y = x^3 + 1\), \(x = 1\), and \(\triangle{x} = -0.5\)
            \item \(y = x^3 + 1\), \(x = -1\), and \(\triangle{x} = 0.1\)
        \end{enumerate}

        \item Find \(dy\).
        \begin{enumerate}[label*=\arabic*.]
            \item \(y = {(3x^2 - 2x + 1)}^3\)
            \item \(y = \frac{3x}{x^2 + 2}\)
            \item \(y = x^2\sqrt[]{2x + 3}\)
            \item \(y = \sqrt[]{4 - x^2}\)
            \item \(y = \frac{2 + \cos{x}}{2 - \sin{x}}\)
            \item \(y = \tan^2{x} \sec^2{x}\)
        \end{enumerate}

        \item Solve the following problems.
        \begin{enumerate}[label*=\arabic*.]
            \item The measurement of an edge of a cube is found to be 15 cm with a possible error of 0.01 cm. Use differentials to find the approximate error in computing from this measurement: (a) the volume; (b) the area of one of the faces.
            \item An open cylindrical tank is to have an outside coating of thickness 2 cm. If the inner radius is 6 m and the altitude is 10 m, find by differentials the approximate amount of coating material to be used.
            \item A burn on a person's skin is in the shape of a circle. Use differentials to find the approximate decrease in the area of the burn when the radius decreases from 1 cm to 0.8 cm.
            \item A tumor in a person's body is spherical in shape. Use differentials to find the approximate increase in the volume of the tumor when the radius increases from 1.5 cm to 1.6 cm.
        \end{enumerate}

    \end{enumerate}

\newpage\subsection*{Answer Key}

    \begin{enumerate}
        
        \item Find \(dy\) and \(\triangle{y}\) for the given values of \(x\) and \(\triangle{x}\).
        \begin{enumerate}[label*=\arabic*.]
            \item \(dy = 2\), \(\triangle{y} = 2.25\)
            \item \(dy = 6\), \(\triangle{y} = 7.625\)
            \item \(dy = \frac{1}{12} \approx 0.083\), \(\triangle{y} = \sqrt[3]{9} - 2 \approx 0.080\)
            \item \(dy = 0.25\), \(\triangle{y} = \sqrt[]{5} - \sqrt[]{4} \approx 0.236\)
        \end{enumerate}

        \item Find (a) \(\triangle{y}\); (b) \(dy\); (c) \(\triangle{y} - dy\).
        \begin{enumerate}[label*=\arabic*.]
            \item (a) 0.0309, (b) 0.03, (c) 0.0009
            \item (a) -0.0996, (b) -0.1, (c) 0.0004
            \item (a) \(\frac{1}{42} \approx 0.0238\), (b) \(\frac{1}{40} = 0.025\), (c) \(-\frac{1}{840} \approx -0.0012\)
            \item (a) \(\frac{1}{42} \approx 0.0238\), (b) \(\frac{1}{45} = 0.022\), (c) \(\-frac{1}{630} \approx -0.0016\)
            \item (a) \(-0.875\), (b) \(-1.5\), (c) \(0.625\)
            \item (a) \(0.271\), (b) \(0.3\), (c) \(-0.029\)
        \end{enumerate}

        \item Find \(dy\).
        \begin{enumerate}[label*=\arabic*.]
            \item \(dy = 3{(3x^2 - 2x + 1)}^2(6x - 2)dx\)
            \item \(dy = \frac{3(2 - x^2)}{{(x^2 + 2)}^2}dx\)
            \item \(dy = \frac{x(5x + 6)}{{(2x + 3)}^{1/2}}dx\)
            \item \(dy = \frac{-x}{\sqrt[]{4 - x^2}}dx\)
            \item \(dy = \frac{1 - 2\sin{x} + 2\cos{x}}{{(2 - \sin{x})}^2}dx\)
            \item \(dy = 2 \tan{x} \sec^2{x}(2\tan^2{x} + 1)dx\)
        \end{enumerate}

        \item Solve the following problems.
        \begin{enumerate}[label*=\arabic*.]
            \item (a) \(6.75 cm^3\), (b) \(0.3 cm^2\)
            \item \(\frac{12}{5} \pi m^3\)
            \item \(0.4 \pi cm^2\)
            \item \(0.9 \pi cm^3\)
        \end{enumerate}

    \end{enumerate}

\end{document}
\documentclass[12pt]{article}
\usepackage[margin=1in]{geometry}
\usepackage{xcolor, color, soul}
\usepackage{amssymb}
\usepackage{enumitem}
\usepackage{booktabs}
\usepackage{amsmath}
\usepackage{array}
\usepackage{tabularx}

\sethlcolor{yellow}

\newcommand{\class}{Integral Calculus for Computer Science Students}
\newcommand{\datewritten}{Term 1, Fall '23}
\newcommand{\instructor}{Course Instructor: Noel T. Fortun}
\newcommand{\notes}{A.L. Maagma}
\newcommand{\follow}{\bigskip\noindent}
\newcommand{\spaces}{\quad\quad\quad}
\newcommand{\spacee}{\quad}
\newcommand{\point}{\,\cdot\,}

\newcommand{\minus}{-}
\newcommand{\block}[1]{\[{#1}\]}
\newcommand{\inline}[1]{\({#1}\)}
\newcommand{\proving}[1]{\begin{align*}{#1}\end{align*}}
\newcommand{\numbers}[1]{\begin{enumerate}[label*=\arabic*.]{#1}\end{enumerate}}
\newcommand{\enclose}[1]{\fbox{\parbox{\dimexpr\linewidth-2\fboxsep-2\fboxrule}{{#1}}}}

\newenvironment{conditions}
{\par\vspace{\abovedisplayskip}\noindent\begin{tabular}{>{$}l<{$} @{${}={}$} l}}
{\end{tabular}\par\vspace*{\belowdisplayskip}}

% ---------------- Start Document ----------------

\begin{document}
\pagestyle{plain}
\thispagestyle{empty}

% -------------------- Header --------------------

\noindent
\begin{tabular*}{\textwidth}{l @{\extracolsep{\fill}} r @{\extracolsep{6pt}} l}
    \textbf{\class} && \textbf{Date: \datewritten} \\
    \textbf{\instructor} && \textbf{Notes: \notes} \\
\end{tabular*}\\
\rule[2ex]{\textwidth}{2pt}

% ------------- Topic: Differentials -------------

\section{The Differentials}

\enclose{
    \textbf{DEFINITION.}
    These provide us with a way of \hl{estimating the amount a function changes as a result of a small change in input values}.
    The equation \block{\triangle{y} \approx{} dy} can and will only be considered if \inline{\triangle{x}} is ``close enough''.
    The approximation of the equation becomes better as \inline{\triangle{x}} becomes smaller.
    
    \follow\textbf{NOTE.}
    We equate \inline{\triangle{x} = dx}.
}

\subsection{The Differential of the Independent Variable}

    \enclose{
        \textbf{DEFINITION.}
        If the function \textit{f} is definited by the equation \inline{y = f (x)}, then the \hl{differential of y}, denoted by \textit{dy}, is given by \block{dy = f' (x) \, dx \longrightarrow{} f' (x) = \frac{dy}{dx}} where \textit{x} is any number in the domain of \textit{f'} and \inline{\triangle{x}} is an arbitrary increment of \textit{x}.
    }

    \follow\textbf{EXAMPLE 1.1.1.}
    Find \textit{dy} for \inline{y = (x^3 + 5x \minus{} 1)^{2023}}.
    \block{f' (x) = 2023 (x^3 + 5x \minus{} 1)^{2022} (3x^2 + 5)}
    \block{\therefore{} dy = 2023(x^3 + 5x \minus{} 1) (3x^2 + 5x) \, dx}

    \follow\textbf{EXAMPLE 1.1.2.}
    Find the differential \textit{dy} of the function \inline{y = 4x^2 + x + 3}.
    \block{(8x + 1) \, dx}

    \follow\textbf{EXAMPLE 1.1.3.}
    Find the differential \textit{dy} of the function \inline{y = \cos(x)}.
    \block{dy = \minus{} \sin(x) \, dx}

    \newpage\follow\textbf{NOTE.}
    The following figures represent the corresponding derivatives of trigonometric identities or functions.
    
    % TODO: Center Table
    \follow\begin{tabular}{ccc}
        \toprule \inline{f (x)} && \inline{f' (x)} \\
        \midrule \inline{\sin(x)} && \inline{\cos(x)} \\ 
        \midrule \inline{\cos(x)} && \inline{\minus{} \sin(x)} \\
        \midrule \inline{\tan(x)} && \inline{\sec^2 (x)} \\
        \midrule \inline{\cot(x)} && \inline{\minus{} \csc^2 (x)} \\
        \midrule \inline{\sec(x)} && \inline{\sec(x) \tan(x)} \\
        \midrule \inline{\csc(x)} && \inline{\minus{} \csc(x) \cot(x)} \\
    \bottomrule\end{tabular}

    \follow\textbf{EXAMPLE 1.1.4.}
    Compare the values of \inline{\triangle{y}} and \textit{dy} if \inline{y = f (x) = x^3 + x^2 \minus{} 2x + 1} and \textit{x} changes (a) from 2 to 2.05 and (b) from 2 to 2.01.
    
    \begin{align*}
        x &= 2                  & \triangle{y} &= f(x + \triangle{x}) - f(x)& dy &= f'(x) \, dx \\
        x + \triangle{x} &= 2.05& &= f(2.05) - f(2)                         & &= (3x^2 + 2x \minus 2) \, dx \\
        \triangle{x} &= 0.05    & &= 0.7176                                 & &= (3{(2)}^2 + 2(2) \minus 2)(0.05) \\
        dx &= 0.05              &&                                          & &= 0.7
    \end{align*}

    \follow\textbf{NOTE.}
    The final equation utilized our solution at \inline{x = 2}, \inline{\triangle{x} = dx = 0.05}.
    Observe the approximation of \inline{\triangle{y} \approx{} dy} becomes better as \inline{\triangle{x}} becomes smaller.

\newpage\subsection*{Practice Exercises}

    \begin{enumerate}
        \item Find \textit{dy} and \inline{\triangle{y}} for the given values of \textit{x} and \inline{\triangle{x}}.
        \numbers{
            \item \inline{y = x^2}, \inline{x = 2}, and \inline{\triangle{x} = 0.5}
            \item \inline{y = x^3}, \inline{x = 2}, and \inline{\triangle{x} = 0.5}
            \item \inline{y = \sqrt[3]{x}}, \inline{x = 8}, and \inline{\triangle{x} = 1}
            \item \inline{ y = \sqrt{x}}, \inline{x = 4}, and \inline{\triangle{x} = 1}
        }
        
        \item Find (a) \inline{\triangle{y}}, (b) \textit{dy}, (c) \inline{\triangle{y} \minus{} dy}.
        \numbers{
            \item \inline{y = x^2 \minus{} 3x}, \inline{x = 2}, and \inline{\triangle{x} = 0.03}
            \item \inline{y = x^2 \minus{} 3x}, \inline{x = \minus{} 1}, and \inline{\triangle{x} = 0.02}
            \item \inline{y = \frac{1}{x}}, \inline{x = \minus{} 2}, and \inline{\triangle{x} = \minus{} 0.1}
            \item \inline{y = \frac{1}{x}}, \inline{x = 3}, and \inline{\triangle{x} = \minus{} 0.2}
            \item \inline{y = x^3 + 1}, \inline{x = 1}, and \inline{\triangle{x} = \minus{} 0.5}
            \item \inline{y = x^3 + 1}, \inline{x = \minus{} 1}, and \inline{\triangle{x} = 0.1}
        }

        \item Find \inline{dy}.
        \numbers{
            \item \inline{y = {(3x^2 \minus{} 2x + 1)}^3}
            \item \inline{y = \frac{3x}{x^2 + 2}}
            \item \inline{y = x^2 \, \sqrt{2x + 3}}
            \item \inline{y = \sqrt{4 \minus{} x^2}}
            \item \inline{y = \frac{2 + \cos(x)}{2 \minus{} \sin(x)}}
            \item \inline{y = \tan^2(x) \sec^2(x)}
        }

        \item Solve the following problems.
        \numbers{
            \item The measurement of an edge of a cube is found to be 15 cm with a possible error of 0.01 cm. Use differentials to find the approximate error in computing from this measurement: (a) the volume; (b) the area of one of the faces.
            \item An open cylindrical tank is to have an outside coating of thickness 2 cm. If the inner radius is 6 m and the altitude is 10 m, find by differentials the approximate amount of coating material to be used.
            \item A burn on a person's skin is in the shape of a circle. Use differentials to find the approximate decrease in the area of the burn when the radius decreases from 1 cm to 0.8 cm.
            \item A tumor in a person's body is spherical in shape. Use differentials to find the approximate increase in the volume of the tumor when the radius increases from 1.5 cm to 1.6 cm.
        }
    \end{enumerate}

\newpage\subsection*{Answer Key}

\begin{enumerate}
    \item Find \textit{dy} and \inline{\triangle{y}} for the given values of \textit{x} and \inline{\triangle{x}}.
    \numbers{
        \item \inline{dy = 2}, \inline{\triangle{y} = 2.25}
        \item \inline{dy = 6}, \inline{\triangle{y} = 7.625}
        \item \inline{dy = \frac{1}{12} \approx{} 0.083}, \inline{\triangle{y} = \sqrt[3]{9} \minus{} 2 \approx{} 0.080}
        \item \inline{dy = 0.25}, \inline{\triangle{y} = \sqrt{5} \minus{} \sqrt{4} \approx{} 0.236}
    }
    
    \item Find (a) \inline{\triangle{y}}, (b) \textit{dy}, (c) \inline{\triangle{y} \minus{} dy}.
    \numbers{
        \item (a) 0.0309, (b) 0.03, (c) 0.0009
        \item (a) -0.0996, (b) -0.1, (c) 0.0004
        \item (a) \inline{\frac{1}{42} \approx{} 0.0238}, (b) \inline{\frac{1}{40} = 0.025}, (c) \inline{\minus{} \frac{1}{840} \approx{} \minus{} 0.0012}
        \item (a) \inline{\frac{1}{42} \approx{} 0.0238}, (b) \inline{\frac{1}{45} = 0.022}, (c) \inline{\frac{1}{630} \approx{} \minus{} 0.0016}
        \item (a) \inline{\minus{} 0.875}, (b) \inline{\minus{} 1.5}, (c) 0.625
        \item (a) 0.271, (b) 0.3, (c) \inline{\minus{} 0.029}
    }

    \item Find \inline{dy}.
    \numbers{
        \item \inline{dy = 3{(3x^2 \minus{} 2x + 1)}^2{(6x \minus{} 2)} \, dx}
        \item \inline{dy = \frac{3(2 \minus{} x^2)}{{(x^2 + 2)}^2} \, dx}
        \item \inline{dy = \frac{x (5x + 6)}{({2x + 3})^{1/2}} \, dx}
        \item \inline{dy = \frac{\minus{} x}{\sqrt{4 \minus{} x^2}} \, dx}
        \item \inline{dy = \frac{1 \minus{} 2 \sin(x) + 2 \cos(x)}{{(2 \minus{} \sin(x))}^2} \, dx}
        \item \inline{dy = 2 \tan(x) \sec^2(x) x (2\tan^2(x) + 1) \, dx}
    }

    \item Solve the following problems.
    \numbers{
        \item (a) \inline{6.75 \, cm^3}, (b) \inline{0.3 \, cm^2}
        \item \inline{\frac{12}{5} \pi{} \, m^3}
        \item \inline{0.4 \pi{} \, cm^2}
        \item \inline{0.9 \pi{} \, cm^3}
    }
\end{enumerate}

\newpage\subsection{Error Propagation}

    \enclose{
        \textbf{DEFINITION.}
        In practice, differentials can be used in the estimation of errors propagated by physical measuring devices.
        If the measure value of \textit{x} is used to compute another value \inline{f (x)}, then the difference between \inline{f (x + \triangle{x})} and \textit{f (x)} is the propagated error.
        
        \block{f (x + \triangle{x}) \minus{} f (x) = \triangle{y} \approx{} dy} where:
        
        \begin{conditions}
            x + \triangle{x} & Exact Value \\
            \triangle{x} & Measurement Error \\
            f (x) & Measured Value \\
            \triangle{y} & Propagated Error
        \end{conditions}

        \textbf{NOTE.}
        How do you know if the propagated error is large or small?
        The answer is best given in \textit{relative} terms by comparing \textit{dA} and \textit{A}.
        The ratio is called the \hl{relative error} and further be expressed as \hl{percentage error}.
    }

    \follow\textbf{EXAMPLE 1.2.1}
    A radius of a sphere is to be 3 cm with a possible error of 0.02 cm. (1) Use differentials to approximate the error in calculating the volume. (2) What is the relative error and percentage error?

    \begin{align*}
        && V &= \frac{4}{3} \pi r^3 \\
        dV &= V' \, dr              &&& \frac{dV}{V} &= \frac{\pm 2.2619}{36 \pi} \\
        &= 4 \pi r^2 \, dr          &&& (2.1) &= \pm 0.02 \\
        &= 4 \pi {(3)}^2 (\pm 0.02) &&& (2.2) &= \pm 2\% \\
        (1) &= \pm 2.2619 \, cm^3
    \end{align*}

    \follow\textbf{SUMMARY.}
    The essence of differentials provide us with a way of estimating the amount a function changes as a result of a small change in input values.

    \follow\textbf{REMARK.}
    Although the application of differentials to approximate function values is not very important in the age of technology, differentials are important as a \hl{convinient notational device} for the computation of antiderivatives.

\newpage\section{Antidifferentiation: Indefinite Integration}

\textbf{EXAMPLE 2.0.0}
In order to find a function \textit{F} whose derivative is \inline{F' (x) = 3x^2}, we use our knowledge of derivatives to conclude the following: \block{F (x) = x^3 \; \textrm{since} \; \frac{d}{dx}[x^3] = 3x^2}
The function \textit{F} is considered an antiderivative of \textit{f}.

\follow\enclose{
    \textbf{DEFINITION.}
    A function \textit{F} is \hl{an antiderivative} of \textit{f} of an interval \textit{I} when \block{F' (x) = f (x)} for all \textit{x} in \textit{I}.
    As for the previous example, \inline{F (x) = x^3} is an antiderivative of \inline{f (x) = 3x^2}.

    \follow\textbf{NOTE.}
    An antiderivative of \textit{f} is not unique because of the infinite possible values for the constant \textit{C}.
}

\subsection{General Antiderivative of a Function}

    \enclose{
        \textbf{DEFINITION.}
        If \textit{F} \hl{is an antiderivative of \textit{f}} on an interval \textit{I}, then G \hl{is an antiderivative of \textit{f}} on the interval \textit{I} if and only if \textit{G} is of the form \block{G (x) = F (x) + C} for all \textit{x} in \textit{I}, where \textit{C} is a constant.
    }

    \follow\enclose{
        \textbf{DEFINITION.}
        The operation of finding the antiderivatves of a function is called \hl{antidifferentiation}, or indefinite integration, and is denoted by an integral sign \inline{\int{}}. 
        Additionally, the equation below states that when we antidifferentiate the differential of a function, we obtain the function plus an arbitrary constant. 
        \block{\int{} f (x) \, dx = F (x) + C}
        \begin{conditions}
            \int{} & Integration Symbol \\
            f (x) & Integrand \\
            dx & Differential of X \\
            F (x) & One Antiderivative \\
            C & Constant of Integration
        \end{conditions}
        
        \textbf{NOTE.}
        The expression \inline{\int{} f (x) \, dx} is read as ``the antiderivative of f with respect to x''. The differential \textit{dx} serves to identify \textit{x} as the variable of integration. The term \hl{indefinite integral} is a synonym for antiderivative.
    }

\newpage\subsection{Basic Integration Rules}

    \bigskip\begin{center}\begin{tabular}{ccc}
        \toprule\ \textbf{Differentiation Formula} && \textbf{Integration Formula} \\
        \midrule\ \inline{\frac{d}{dx}[C] = 0} && \inline{\int{} 0 \, dx = C} \\
        \midrule\ \inline{\frac{d}{dx}[kx] = k} && \inline{\int{} k \, dx = kx + C} \\
        \midrule\ \inline{\frac{d}{dx}[kf (x)] = kf' (x)} && \inline{\int{} kf (x) \, dx = k \int{} f (x) \, dx} \\
        \midrule\ \inline{\frac{d}{dx}[f (x) \pm{} g (x)] = f' (x) \pm{} g ' (x)} && \inline{\int{} [f (x) \pm{} g (x)] \, dx = \int{} f (x) \, dx \pm{} \int{} g (x) \, dx} \\ 
        \midrule\ \inline{\frac{d}{dx}[x^n] = {nx}^{n \minus{} 1}} && \inline{\int{} x^n \, dx = \frac{x^{n + 1}}{n + 1} + C}, \inline{n \neq{} \minus{} 1} \\
        \midrule\ \inline{\frac{d}{dx}[\sin{x}] = \cos{x}} && \inline{\int{} \cos{x} \, dx = \sin{x} + C} \\
        \midrule\ \inline{\frac{d}{dx}[\cos{x}] = \minus{} \sin{x}} && \inline{\int{} \sin{x} \, dx = \minus{} \cos{x} + C} \\
        \midrule\ \inline{\frac{d}{dx}[\tan{x}] = \sec^2{x}} && \inline{\int{} sec^2{x} \, dx = \tan{x} + C} \\
        \midrule\ \inline{\frac{d}{dx}[\sec{x}] = \sec{x} \tan{x}} && \inline{\int{} \sec{x} \tan{x} \, dx = \sec{x} + C} \\
        \midrule\ \inline{\frac{d}{dx}[\cot{x}] = \minus{} \csc^2{x}} && \inline{\int{} \csc^2{x} \, dx = \minus{} \cot{x} + C} \\
        \midrule\ \inline{\frac{d}{dx}[\csc{x}] = \minus{} \csc{x} \cot{x}} && \inline{\int{} \csc{x} \cot{x} \, dx = \minus{} \csc{x} + C} \\
    \bottomrule\end{tabular}\end{center}

\subsection{Power Rule of Integrals}

    \enclose{
        \textbf{DEFINITION.}
        As for the power rule of integrals, as long as \inline{n \neq{} 1},
        \block{\int{} x^n \, dx = \frac{x^{n + 1}}{n + 1} + C}
    }

    \follow\textbf{EXAMPLE 2.3.1.}
    Find the antiderivative of \inline{\int{} x^5 \, dx}.
    \proving{
        &= \frac{x^6}{6} + C
    }

    \follow\textbf{EXAMPLE 2.3.2.}
    Find the antiderivative of \inline{\int{} \sqrt{x} \, dx}.
    \proving{
        &= \int{} x^{1/2} \, dx \\
        &= \frac{x^{\frac{1}{2} + 1}}{\frac{1}{2} + 1} + C \\
        &= \frac{x^{\frac{3}{2}}}{\frac{3}{2}} + C \\
        &= \frac{2 x^{\frac{3}{2}}}{3} + C
    }

    \newpage\follow\textbf{EXAMPLE 2.3.3.}
    Find the antiderivative of \inline{\int{} 3x \, dx}.
    \proving{
        &= 3 \int{} x \, dx \\
        &= 3 \point{} \frac{x^2}{2} \\
        &= \frac{3x^2}{2} + C
    }

    \follow\textbf{EXAMPLE 2.3.4.}
    Find the antiderivative of \inline{\int{} \frac{1}{x^3} \, dx}.
    \proving{
        &= \minus{} \frac{1}{2x^2} + C
    }

    \follow\textbf{EXAMPLE 2.3.5.}
    Find the antiderivative of \inline{\int{} 2 \sin{x} \, dx}.
    \proving{
        &= 2 \int{} \sin{x} \, dx \\
        &= 2 \point{} \minus{} \cos{x} \\
        &= \minus{} 2 \cos{x} + C
    }

    \follow\textbf{EXAMPLE 2.3.6.}
    Find the antiderivative of \inline{\int{} (3x^4 \minus{} 5x^2 + x) \, dx}.
    \proving{
        &= 3 \int{} x^4 \, dx \minus{} 5 \int{} x^2 \, dx + \int{} x \, dx \\
        &= \frac{3x^5}{5} \minus{} \frac{5x^3}{3} + \frac{2x^2}{2} + C
    }

    \follow\textbf{EXAMPLE 2.3.7.}
    Find the antiderivative of \inline{\int{} \frac{x + 1}{\sqrt{x}} \, dx}.
    \proving{
        && &= \int{} \left(\frac{x}{\sqrt{x}} + \frac{1}{\sqrt{x}}\right) \, dx \\
        && &= \int{} \left(\sqrt{x} + \frac{1}{\sqrt{x}}\right) \, dx \\
        &= \int{} \sqrt{x} \, dx &&& &= \int{} \frac{1}{\sqrt{x}} \, dx \\
        &= \int{} x^{\frac{1}{2}} \, dx &&& &= \int{} x^{\minus{} \frac{1}{2}} \, dx \\
        &= \frac{2 \sqrt{x^3}}{3} &&& &= 2 \sqrt{x} \\
        && &= \frac{2 \sqrt{x^3}}{3} + 2 \sqrt{x} \\
        && &= \frac{2 \sqrt{x} (x + 3)}{3} + C
    }

    \newpage\follow\textbf{NOTE.}
    The following are reciprocal and pythagorean identities for trigonometric integrals.
    \proving{
        \cot{x} &= \frac{1}{\tan{x}} = \frac{\cos{x}}{\sin{x}} & \sin^2{x} &+ \cos^2{x} = 1 \\
        \csc{x} &= \frac{1}{\sin{x}} & \sec^2{x} &= 1 + \tan^2{x} \\
        \sec{x} &= \frac{1}{\cos{x}} & \csc^2{x} &= 1 + \cot^2{x}
    }
    

    \follow\textbf{EXAMPLE 2.3.8.}
    Find the antiderivative of \inline{\int{} \frac{\sin{x}}{\cos^2{x}} \, dx}.
    \proving{
        &= \frac{1}{\cos{x}} + C \\
        &= \sec{x} + C
    }

\end{document}